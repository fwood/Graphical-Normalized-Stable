\documentclass{article}
\usepackage{nips10submit_e,times}


%\documentstyle[nips07submit_09,times]{article}
\usepackage[square,numbers]{natbib}
%\usepackage{amsmath, epsfig}
%\usepackage{amsfonts}
%\usepackage{subfigure}
%\usepackage{graphicx}
%\usepackage{amsfonts}
\usepackage{algorithm}
\usepackage{algorithmic}
%\usepackage{easybmat}
%\usepackage{footmisc}
%\usepackage{lscape}
%\usepackage{subfigure}
%\usepackage{float}
\usepackage{amsmath}
\usepackage{amsthm}
\usepackage{amsfonts}
\usepackage{amssymb}
%\usepackage{aistats2012}
\usepackage{epsfig}
\usepackage{amsmath}
\usepackage{amsthm}
\usepackage{amsfonts}
\usepackage{amssymb}
\usepackage{framed}
\usepackage{natbib}
\usepackage{subfig}
\usepackage{xspace}
\usepackage[normalem]{ulem}
\usepackage[usenames,dvipsnames]{color}
\usepackage{balance}
\renewcommand\algorithmiccomment[1]{// \textit{#1}}
%
\newcommand{\ignore}[1]{}
\newcommand{\comment}[1]{}
\DeclareMathOperator*{\argmax}{arg\,max}

\long\def\comment#1{}

\newcommand{\eq}{\begin{equation*}}
\newcommand{\en}{\end{equation*}}
\newcommand{\eqa}{\begin{eqnarray*}}
\newcommand{\ena}{\end{eqnarray*}}
\newcommand{\eqn}{\begin{equation}}
\newcommand{\enn}{\end{equation}}
\newcommand{\eqan}{\begin{eqnarray}}
\newcommand{\enan}{\end{eqnarray}}

\newcommand{\sinead}[1]{\textcolor{red}{\textsf{\emph{\textbf{\textcolor{red}{#1}}}}}}
\newcommand{\frank}[1]{\textcolor{red}{\textsf{\emph{\textbf{\textcolor{blue}{#1}}}}}}
\newcommand{\cosma}[1]{\textcolor{red}{\textsf{\emph{\textbf{\textcolor{green}{#1}}}}}}

\newcommand{\NS}{\ensuremath{\mathcal{NS}}}
\newcommand{\PY}{\ensuremath{\mathcal{PY}}}
\newcommand{\DP}{\ensuremath{\mathcal{DP}}}
\newcommand{\ngram}{\ensuremath{n}-gram }

\title{Graphical Normalized Stable Processes }
%Supervised Topic Modeling in Clinical Text}

\author{
Frank Wood\\
Columbia University\\
New York, NY 10027, USA \\
\texttt{fwood@stat.columbia.edu}
\and
\bf Sinead Williamson\\
 Carnegie Mellon University\\
Pittsburgh, PA 15213, USA\\
\texttt{sinead+@cs.cmu.edu} 
%\texttt{pfau@neurotheory.columbia.edu} 
%\texttt{\{bartlett,fwood\}@stat.columbia.edu} 
}



% The \author macro works with any number of authors. There are two commands
% used to separate the names and addresses of multiple authors: \And and \AND.
%
% Using \And between authors leaves it to \LaTeX{} to determine where to break
% the lines. Using \AND forces a linebreak at that point. So, if \LaTeX{}
% puts 3 of 4 authors names on the first line, and the last on the second
% line, try using \AND instead of \And before the third author name.

\newcommand{\fix}{\marginpar{FIX}}
\newcommand{\new}{\marginpar{NEW}}
\newcommand{\X}{\mathcal{X}}


\nipsfinalcopy

\begin{document}

\maketitle

\begin{abstract}
\input{abstract}
\end{abstract}

\section{Introduction}
\label{sec:introduction}
Hierarchical models for regularizing conditional density estimates are central to most applications in statistics and machine learning.  The sharing of statistical strength by tying together distributions is imperative for inference in models with a large number of parameters, especially when the number of parameters is larger than the number of observations.  

In this work we are motivated by the goal of maximizing sharing of statistical strength via learning of back-off paths in a dense, hierarchical graphical models.  One intuition for this comes from natural language modeling applications.  Consider the distribution of words that follow ``She ate 2 pieces of  $\ldots$''  In this case we will call ``she ate 2 pieces of'' the ``context.''  Recognize that the context indexes a conditional distribution (over words).  Traditional \ngram language modeling approaches to estimating the conditional distribution over words that follow this context would look for the string ``she ate 2 pieces of'' in the corpus and form a counting estimate of the conditional distribution from that.  Of course, this estimate will be extremely noisy in long contexts, and, more generally, any time the underlying stochastic process produces observations in a large vocabulary or has long-range dependencies.   So, what is usually done is to hierarchically relate the distribution over words that follow the context ``She ate 2 pieces of $\ldots$'' to the distribution over words that follow ``ate 2 pieces of $\ldots$'' and that to ``2 pieces of $\ldots$'' and so on.  Of course, in this case, it is clear that we would rather generalize via a different path.  In fact, we'd much rather generalize via the path ``She ate 2 pieces of $\ldots$'' is more specific than ``She ate * pieces of $\ldots$'' is more specific than ``* ate * pieces of $\ldots$'' is more specific than ``* ate * * of \ldots'' is more specific than ``* ate * * * \ldots.''  Of course, it goes without saying, each and every one of these contexts generalizes in many, many different ways.  More generally one can think about the context as a bag of features and the generalization paths accessible from that bag as functions that map specific contexts to less specific contexts in which one more more of the values of the features is disregarded.

Another example comes from modeling image statistics.  Consider filling in the central pixel in a $5\times5$ binary image patch.  Here we would like to estimate the conditional distribution of the pixel given its surround, however, finding patches that exactly match the surround is extremely unlikely even in highly regular domains.  Also, very clearly here, more back-off paths than just ignoring the value of a single pixel are essential.

Note that hidden or latent variables can be interpreted and utilized as features in this set up.  


%\section{Background}



\section{A completely random measure/L\'{e}vy process view of chains and networks of normalized stable processes}

\subsection{Completely random measures}
Let $(X,\mathcal{X})$ be some observable space, and $(M,\mathcal{M})$ be the space of measures on that space such that, for any measure $\rho\in\mathcal{M}$, the measures assigned to disjoint subsets $A_1,A_2,\dots \in\mathcal{X}$ are independent. A \emph{completely random measure} (CRM) is a probability distribution over the space $(M,\mathcal{M})$ of such measures.

A completely random measure is an example of an \emph{infinitely divisible} distribution. We say a distribution $\mu$ is infinitely divisible if, for any $X\sim\mu$, and any $n\in\mathbb{n}$, we can write $X = \sum_{i=1}^{n}X^{(i)}$, wherethe $X^{(i)}$ are distributed i.i.d. according to some distribution $\mu^{(n)}$. The independence assumption implicit in the CRM definition ensures that this is the case.

Infinitely divisible distributions can be uniquely defined using the L\'{e}vy-Khintchine representation. The L\'{e}vy-Khintchine representation describes the characteristic function of such distribtions in terms of a measure $\tilde{nu}$ (which governs any discrete random part of the distribution), a drift vector $b$ (which governs any deterministic part of the distribution), and a Gaussian covariance matrix $A$ (which governs any continuous part of the distribution). In the case of completely random measures with no deterministic component,\footnote{A deterministic measure satisfies the requirement of a completely random measure; however we can generally treat any deterministic component of our distribution separately from the random component} the L\'{e}vy-Khintchine representation reduces to the following form:

\begin{equation*}
\mathbb{E}\bigg[e^{-t\mu(A)}\bigg] = \exp\bigg\{ \int_{z=0}^\infty\int_{x\in A}(1-e^{-tz}\tilde{\nu}(dx,\,dz)\bigg\}\, ,
\end{equation*}
where $\tilde{\nu}(dx,\,dz)$ is a positive measure on $X \times \mathbb{R}_+$ that satisfies certain conditions (see eg Sato). In the cases considered herein, and indeed most cases in the literature, we can decompose  $\tilde{\nu}(dx,\,dz) = H(dx)\nu(dz)$\footnote{Cases where this does not hold are often referred to as inhomogeneous or additive CRMs}. We will refer to $H(dx)$ as the base measure, and $\nu(dz)$ as the L\'{e}vy measure.\footnote{Note that the measure $\tilde{\nu}$ is sometimes referred to as the L\'{e}vy measure} The base measure controls the location of the atoms in a CRM-distributed measure; the L\'{e}vy measure controls their size.

Completely random measures on $\mathbb{R_+}$ are related to a class of L\'{e}vy processes known as \emph{subordinators}. A subordinator is a L\'{e}vy process whose sample paths are strictly increasing. This increasing nature means that subordinators are characterized by the same form of the L\'{e}vy-Khintchine representation as completely random measures, although the base measure is generally assumed to be Lebesgue measure. The magnitudes and locations of the jumps in a subordinator with L\'{e}vy measure $\nu$ correspond to the magnitudes and locations of the atoms in a completely random measure with L\'{e}vy measure $\nu$ and Lebesgue base measure. 

Subordinators can be used as operators on L\'{e}vy processes.  Let $\{X(t),t>0\}$ be a L\'{e}vy process, and let $\{Z(t),t>0\}$ be a subordinator. Then the stochastic process defined by $Y(t) = X(Z(t))$ is said to be \emph{subordinated} to $X$ via $Z$. $Y(t)$ is again a L\'{e}vy process, with and its L\'{e}vy measure, drift and Gaussian covariance matrix can be obtained from the corresponding values for $X$ and $Z$ as described in (Sato p 198). If $X$ is again a subordinator, it follows that $Y$ is also a subordinator. 

Samples from completely random measures can be normalized to obtain distributions over probability measures (Kingman:NRMs). While this is possible for any CRM, in most cases it is not practical since the distribution of the normalized atom sizes depends on the normalizing constant. As a result, it is not possible to sample a finite number of atoms from the normalized random measure. However, there are two cases where the normalizing constant is independent of the normalized atom sizes: The gamma process, and the stable process.


The gamma process is the completely random measure with L\'{e}vy measure $\nu(dz) = \theta z^{-1}e^{-cz} dz$. If $\gamma(\cdot)$ is a Gamma process on some space $X$, then the normalized random measure $\gamma(\cdot)/\gamma(X)$ is a Dirichlet process, and the distribution of the associated mass partition of $X$ is called the Poisson-Dirichlet distribution with parameters $(0,\theta)$.

The stable process is the completely random measure with L\'{e}vy measure ...


\section{Model}
\label{sec:model}
The most general way of writing down what I want is the following.  Let  $G_{\mathbb{X}}$ be a conditional distribution indexed by context/ observed/unobserved feature set $\mathbb{X}$.  For now, let contexts be members of the set of all $D$ dimensional binary vectors $\mathbb{X} \in \{0,1\}^D$.  Let $\mathrm{pa}(\mathbb{X}) = \{ x \in \{0,1\}^{D-1} \;\mbox{s.t.}\; x(d)= \mathbb{X}(i) \,\forall\, i \}$ \frank{this notation is totally f'ed up but I can't be bothered right now} be a function that maps a context to all of it's $D-1$ dimensional generalizations.  In words,  $\mathrm{pa}(\mathbb{X})$ is the set of all contexts that are equal in value except one entry.

\frank{My dream here is {\em all} generalizations.  Note, I have been intentionally vague about whether or not the value is \underline{marginalized out} (i.e.~replaced with a wildcard) or  \underline{deleted}, i.e.~treated as if not actually there.  This is because I'm not sure which is more important or which I want for sure.  Actually, in conditioning is there really a difference?}

\frank{Also, I'm bothered while writing this that what I'm aiming for might simply be structure learning for an automata of fixed architecture.  Much better then to ensure that latent variables get mixed into the set of features.}

OK, here goes.  

Let 

\eqa
G_{\epsilon} &\sim& \NS(\alpha_0,\mathbb{U}) \\
G_{\mathbb{X}} &\sim& \NS(\alpha_{|\mathbb{X}|}, \sum_{x \in \mathrm{pa}{(\mathbb{X})}}\lambda_x G_x)\\
x|\mathbb{X} &\sim& G_{\mathbb{X}}
\ena

\subsection{Graphical Normalized Stable}

Let
$\NS(\alpha,H)$ be a normalized stable process (e.g. your \PY) with base measure $H$ and
discount parameter $\alpha$.)

if:
\eqa
H &\sim& \NS(\alpha_1, \lambda_A A + \lambda_B B + \lambda_C C)\\
G &\sim& \NS(\alpha_2, H)
\ena

then does:
\eqn G \sim \NS(\alpha^*, \gamma_A A + \gamma_B B + \gamma_C C) \enn ?

I have a couple of questions/comments:

1. Are $A$, $B$ and $C$ assumed to be observed nodes? If so, then we can
treat  \eqn (\lambda_A A + \lambda_B B + \lambda_C C) \enn as any old base measure,
which means that:

\eqn G\sim\NS(\alpha_1 \alpha_2, \lambda_A A + \lambda_B B + \lambda_C C) \enn

2. Are $A$, $B$ and $C$ assumed to be unobserved as well? If so, do we have
something like \eqn A \sim \NS(\alpha_A, \Omega_A) \enn, etc? ie are they all samples
from some \PY? If so, then:

\eqn (\lambda_A A + \lambda_B B + \lambda_C C) \sim \NS(\alpha^*, \Omega_A + \Omega_B
+ \Omega_C) \enn,
iff the following two conditions hold:

\eqa
1. && \alpha_A = \alpha_B = \alpha_C = \alpha^* \\
2. && \lambda_k \sim \mbox{alpha-stable}(\alpha*)  \leftarrow  \mbox{alpha-stable distribution}
\ena

Unfortunately, you can only add stable processes to get a stable
process if they have the same $\alpha$. Which is somewhat limiting, both
in what you have to instantiate (basically, all your nodes in the sum
have to have the same distance to their most recent observed parent),
and in allowing a random number of nodes in the summation (because you
have to have a symmetric distribution over the weights $\lambda$ -- with
\DP s, you can have a skewed distribution).

If this holds, then:
\eqa
G|\Omega_A, \Omega_B, \Omega_C &\sim& \NS(\alpha* \alpha_1, \Omega_A + \Omega_B + \Omega_C)\\
H|\Omega_A, \Omega_B, \Omega_C &\sim& \NS(\alpha* \alpha_1 \alpha_2, \Omega_A +
\Omega_B + \Omega_C)
\ena

Do either of those give you what you need? If not, can you explain
again what you're after?

\subsection{Interesting result for compression?}

Question: is this of interest too?  And what does it ``given $G_2$'' mean here?  \frank{It means, in the sequence memoizer sense, that $G_2$ isn't marginalized out of the representation and that, in the standard HDP sense, we have (whilst sampling) draws from it (CRF-style).}

\eqan
G_0 & \sim& \NS(\\alpha_1, H) \\
G_1 & \sim&  \NS(\\alpha_2, G_0) \\
G_2 & \sim&  \NS(\\alpha_2, G_0) \\
G_1 | G_2, G_0 &\sim&  ?
\enan

\section{Inference}
\label{sec:inference}
\input{inference}

\section{Related Work}

\label{sec:related_work} 
\input{related_work}

\section{Experiments}

\label{sec:experiments}
\input{experiments}


\section{Discussion}
\label{sec:discussion}
\input{discussion}


\begin{small} \bibliographystyle{plainnat} \bibliographystyle{plainnat}
\bibliographystyle{plainnat}
\bibliography{refs}


\end{small} 

\end{document}

